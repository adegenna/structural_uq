\title{Perturbation Theory for Structural UQ}

% \author{
%         Anthony M. DeGennaro, Associate Computational Scientist, CSI
% }
\date{}

\documentclass[11pt]{article}
\usepackage[margin=1in]{geometry}
\usepackage{amsmath, amsfonts, bm, bbm, graphicx, caption, subcaption}
\usepackage{titling}
\usepackage{lipsum}
\setlength{\droptitle}{-0.75in}   % Adjust title margin
%% \pagenumbering{gobble}

\begin{document}
\newcommand*{\vertbar}{\rule[-1ex]{0.5pt}{2.5ex}}
\newcommand*{\horzbar}{\rule[.5ex]{2.5ex}{0.5pt}}
\maketitle

\vspace*{-0.75in}

\section{Review: Perturbation Theory}

The goal is to help mitigate the problem of structural UQ by using perturbation theory. 
We review the basics of perturbation theory here~\cite{khalil}.
Consider the equation:

\begin{equation}
        \label{eq:dynsys}
        \begin{aligned}
                \dot{u} &= f( u , t , \epsilon )  \\
                u &\in U \subset \mathbb{R}^n \;\;\; , \;\;\; t \in [t_0,t_f] \;\;\; , \;\;\; \epsilon \in [-\epsilon_0 , \epsilon_0].
        \end{aligned}
\end{equation}

\noindent Assume our initial conditions depend smoothly and deterministically on $\epsilon$:

\begin{equation}
        \label{eq:ics}
        u(t_0) = \eta(\epsilon) \;\;\; .
\end{equation}

\noindent Assume also that we have an unperturbed solution $u_0(t) = u(t,0)$.

\noindent Our goal is to quantify Eq.~\ref{eq:dynsys} over the distribution $\epsilon \sim \rho(\epsilon)$. 
The naive route is Monte Carlo sampling. 
However, one use of perturbation theory is to avoid this through separation of variables: we posit this expansion for the solution:

\begin{equation}
        \label{eq:expansion}
        u( t , \epsilon ) \approx \sum_{k=0}^{N-1} u_k(t) \epsilon^k \;\;\; .
\end{equation}

\noindent The fields $u_j(t)$ must be solved for through a system of ODEs obtained by substituting Eq.~\ref{eq:expansion} into Eq.~\ref{eq:dynsys} and matching terms in $\epsilon$:

\begin{equation}
        \label{eq:perteq}
        \begin{aligned}
                \dot{u}_0 &= f( t , u , 0 ) \\
                \dot{u}_k &= A(t) u_k + g_k( t , u_0(t) , \dots , u_{k-1}(t) ) \;\;\; ,
        \end{aligned}
\end{equation}

\noindent where $A(t) = [ \partial f / \partial u ]$ evaluated at $u_0(t)$.

\noindent The advantage for UQ is, we need only solve Eq.~\ref{eq:perteq} {\it once}. 
Then, we simply evaluate Eq.~\ref{eq:expansion} for any value of $\epsilon$ we like.

\section*{}
%\vspace*{-0.75in}
\newpage
\bibliographystyle{plain}
\bibliography{notes.bib}


\end{document}
